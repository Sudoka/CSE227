\section{Context}
As services are often developped around virtualization, vulnerabilities in hypervisors can now
have critical impact on businesses.

\subsection{Impact of hypervisor vulnerabilites}
We assume that the attacker has access to a virtual machine running on a shared
host.
If an attacker is able to find a vulnerabilitie in the hypervisor, it may have
terrible consequence on the other virtual machines :
\begin{description}
\item[Denial of Service] If the attacker is able to crash the hypervisor or part
of it, he could impact all the virtual machine running on top of it.
\item[Data leak] If an attacker is able to read arbitrary location in the memory
or hard drive of the hypervisor, he could steal business critical information.
Even through an attacker can't find a way to execute code he can still have
access of some resources like the hard drive and the attacker can steal
other users information or sensitive data.
\item[Control flow hijacking] This case would be the worst one. If the attacker
is able to execute arbitrary code in the host, he would have access of all
resources in the host, including the ressources of the other virtual machines.
This would allow him to access data, crash the other virtual machines but also
execute more subtle attacks.
\end{description}

These attacks are widely documented in the litterature, the main point is that
vulnerabilities at the hypervisor level may allow an attacker to have access to
the host resources but also to other virtual machines.

\section{Aim}
We think that we may answer the following questions :
\begin{itemize}
\item Are there similarities in the vulnerabilites which were already disclosed ?
\item Are the recent discolsed vulnerabilities exploitable to gain access to host
ressources from the guest ?
\item Can we easily find new vulnerabilities ?
\item How can we detect or prevent them ?
\end{itemize}

\section{Study case}
There are many ways to acces a virtual machine, through other vulnerabilities of
the softwares which are running inside of the virtual machine or through
usage of a public service (e.g. VPS provided by \emph{Rackspace} or instances
provided by \emph{Amazon EC2}).
In our research, we assume that the attacker has access to a virtual machine
running on top of a shared host.
Our study will be focused on two majors software that are VMware Workstation and QEMU-KVM on Linux.


Our research will be divided in three categories:
Detections:

Exploitation:

We will try to exploit one of the vulnerabilities cited before or new ones
detected by as, this is the most important part of our research project because,
if we successfully exploit one vulnerabilitie we will be able to have access to
the host and demonstrate the consequence of that

Defence:

If we exploit a vulnerability with success the next step in our research is to
propose some kind of defense to this problem.

Conclusion

Now a days the virtualization is part of our live and we have to take care of
this new technological


\section{Awaited result}
