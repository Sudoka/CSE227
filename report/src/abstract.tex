\paragraph{Abstract}
With the emergence of cloud computing and server consolidation at large scale,
virtualization has reached an important status in the business development and IT management
of almost all companies.
Usually virtualization is considered as a way to isolate application, for
performance and for security reason.
Vulnerabilities in hypervisors can now have a critical impact and defeat
the assumption that virtualization is a protection.
An application vulnerabilitie coupled to an hypervisor vulnerabilitie could
possibly allow an attacker to take control of hundreds of virtual machine.

Some open source operating system already includes additional security measures
to isolate the different virtual machine, for example the Linux distribution
\emph{Ubuntu}\cite{ubuntu} choosed to add an isolation level by associating a different
\emph{AppArmor}\cite{apparmor} profile with each guest virtual machine running
qemu-kvm.
