\section{Trends}
Since the beginning of the mass adoption of virtualization solutions on x86, a
number of solution attempting to improve security has been seen.
The major improvements come from the hardware manufacturers but some 
security improvement have also been offered by the software side.

\subsection{Software solutions}
Different strategies have been adopted depending on the type of the VMM.

\paragraph{Reducing the Trusted Computing Base (TCB)}
It is received wisdom that a smaller TCB corresponds to more trustworthy code.
Some previous work on the Xen\cite{xen} hypervisor attempted to reduce the
TCB\cite{xen-disaggregation} to
improve security.
This is also something claimed by \emph{VMware}\cite{vmware-footprint} who
compares the size of their hypervisor ESXi (now renamed \emph{VMware vSphere
Hypervisor}) to \emph{Microsoft Hyper-V}.
This is a solution for all types of hypervisors.

\paragraph{Using the operating system functionalities}
Type 2 VMM (VMware Player, VMware Workstation, QEMU, VirtualBox...) run on top
of an other operating system.
This allow the user to isolate the different virtual machines using the
operating system functionnalities.
Some open source operating system already includes additional security measures
to isolate the different virtual machine, for example the Linux distribution
\emph{Ubuntu}\cite{ubuntu} choosed to add an isolation level by associating a different
\emph{AppArmor}\cite{apparmor} profile with each guest virtual machine running
qemu-kvm.

\subsection{Hardware solutions}
Without hardware support, virtualization is achieved by two way :
\begin{itemize}
\item Binary translation, which implies to rewrite every instruction. This is
really expensive in term of performance.
\item Paravirtualization, where the guest OS is modified to call the hypervisor
instead of executing privileged instruction.
\end{itemize}

As seen before, part of the actual work around virtualization security is aimed at
reducing the TCB and hardware assisted virtualization.

\subsubsection{Processor and memory virtualization}
Intel VT avoid

\subsubsection{Device virtualization}
