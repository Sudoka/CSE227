\newpage
\section{Trends}
Since the beginning of the mass adoption of virtualization solutions on x86, a
number of solution attempting to improve security has been seen.
The major improvements come from the hardware manufacturers but some 
security improvement have also been offered by the software side.

\subsection{Software solutions}
Different strategies have been adopted depending on the type of the VMM.

\paragraph{Reducing the Trusted Computing Base (TCB)}
It is received wisdom that a smaller TCB corresponds to more trustworthy code.
Some previous work on the Xen\cite{xen} hypervisor attempted to reduce the
TCB\cite{xen-disaggregation} to
improve security.
This is also something claimed by \emph{VMware}\cite{vmware-footprint} who
compares the size of their hypervisor ESXi (now renamed \emph{VMware vSphere
Hypervisor}) to \emph{Microsoft Hyper-V}.
This is a solution for all types of hypervisors.

\paragraph{Using the operating system functionalities}
Type 2 VMM (VMware Player, VMware Workstation, QEMU, VirtualBox...) run on top
of an other operating system.
This allow the user to isolate the different virtual machines using the
operating system functionnalities.
Some open source operating system already includes additional security measures
to isolate the different virtual machine, for example the Linux distribution
\emph{Ubuntu}\cite{ubuntu} choosed to add an isolation level by associating a different
\emph{AppArmor}\cite{apparmor} profile with each guest virtual machine running
qemu-kvm.

\subsection{Hardware solutions}
Without hardware support, virtualization is achieved by two way :
\begin{itemize}
\item Binary translation, which implies to rewrite every instruction. This is
really expensive in term of performance.
\item Paravirtualization, where the guest OS is modified to call the hypervisor
instead of executing privileged instruction.
\end{itemize}

As seen before, part of the actual work around virtualization security is aimed at
reducing the TCB and hardware assisted virtualization achieve by reducing the
number of functionnalities wich the VMM needs to implement.

\subsubsection{Processor and MMU virtualization}
In 2005, \emph{Intel} introduced a new set of instruction called VT-x rapidly
followed by \emph{AMD} with it's AMD-V technology.
These technology avoided the need of binary translation by executing the guest
OS in a different privilege ring as in paravirtualization.
The solution introduced to avoid rewriting the guest OS code was to allow the
VMM to define the behavior of priviliged instruction in the guest OS.
With these technologie, the VMM is able to configure the processor to produce a
hardware fault trapping into the VMM when the guest OS execute a priviliged
instruction.
This instruction set reduce the amount of code needed.

The adoption was slowed by the necessity to trap into the VMM on each priviliged
instruction causing a big performance penality.
Later, Intel and AMD introduced new instruction sets (Extension Page Table (EPT) and
Rapid Virtualization Indexing (RVI)) to avoid the traps into the VMM on page table
updates.
This avoided the need for the VMM to emulate the MMU.

\subsubsection{Device virtualization}
As seen before, many of the vulnerabilities come from the devices
virtualization.
Again, Intel and AMD introduced two new instruction sets respectivly called
VT-d and AMD-Vi to improve performance and security.
These instruction sets allow the re-mapping of memory for devices using Direct
Memory Access (DMA).
They introduce a I/O Memory Management Unit (IOMMU) on the chipset intended to
map the guest memory to the host memory in the manner of the EPT/RVI.
The IOMMU allows to dedicate a device to a virtual machine and still maintaining the
memory isolation.
These techonologies avoid the need of device emulation or paravirtualized
devices again reducing the TCB.

One main problem with IOMMU is that the physical host needs to have one
set of device (network card, hard disk drive...) for each virtual machine.
Some new specification like the Single Root I/O Virtualization (SR-IOV)
developped by the PCI-SIG (PCI Special Interest Group)
address this problem by allowing a device to simulate multiple devices.
For example, Cisco UCS M81KR network card allow to simulate up to 128 virtual
devices including virtual NIC and HBA.
